\documentclass[12pt,a4paper]{article}
\usepackage{algorithm,algpseudocode}
\usepackage{amsmath,amssymb,amsthm}
\usepackage[english]{babel}
\usepackage[backend=biber]{biblatex}
\usepackage[utf8]{inputenc}		% This has to come before csquotes
\usepackage{csquotes}
\usepackage[T1]{fontenc}		% Use 8-bit CM-like font set
\usepackage{graphicx}
\usepackage{hyperref}
\usepackage{lmodern}
\usepackage{mathtools}

\addbibresource{./ref.bib}

\theoremstyle{definition}
\newtheorem{definition}{Definition}

\newcommand{\myBranchName}{Computer Science and Engineering}
\newcommand{\myCollegeName}{National Institute of Technology, Calicut}
\newcommand{\myDeptName}{Department of Computer Science and Engineering}
\newcommand{\myName}{Arpan Kapoor}
\newcommand{\myRollNo}{B120555CS}
\newcommand{\mySeminarCoordinator}{Vinith R.}
\newcommand{\myTopic}{Dynamic Searchable Symmetric Encryption}

\hypersetup{
	pdftitle={\myTopic{}},
	pdfauthor={\myName{}},
	pdfsubject={Seminar Report},
	pdfkeywords={seminar,report}
}

\begin{document}

\include{title}
\include{certificate}
{\centering \textbf{Acknowledgement}\\[0.25cm]}

I would like to acknowledge the help and guidance received from several people
in preparation of this report.

First and foremost, I would like to thank the course coordinators Lijiya A,
Sameer Mohamed Thahir, Vinith R and Ibrahim A for helping us choose an
appropriate topic for the seminar.

I would like to thank Dr.~Vinod Pathari for his Computer Security course, which
spurred my interest in that domain.

Finally, I would like to thank my family and friends for their continual
support in all my endeavours.

\documentclass[12pt,a4paper]{article}
\usepackage{hyperref,csquotes}
\usepackage[backend=biber]{biblatex}

\addbibresource{ref.bib}

\hypersetup{
	hidelinks,
	pdftitle={Pretty Good Privacy},
	pdfauthor={Arpan Kapoor},
	pdfsubject={Seminar Abstract},
	pdfkeywords={seminar,abstract}
}

\begin{document}

\title{\Huge{Pretty Good Privacy}}
\author{Arpan Kapoor\\
	\href{mailto:arpan_b120555cs@nitc.ac.in}
	{\nolinkurl{arpan_b120555cs@nitc.ac.in}}
}
\date{\today}

\maketitle

\begin{center}
	\textbf{Domain:} Computer Security
\end{center}

\begin{abstract}
With the increasing surveillance efforts on the Internet by various governments,
the demand\slash need for privacy has also escalated. PGP empowers people to
take their privacy in their own hands. It provides security services for
electronic communication and data storage using a combination of strong
public-key and symmetric cryptography. These services include confidentiality,
key management, authentication, and digital signatures.
\end{abstract}

\subsubsection*{Concepts to be covered in seminar:}
\begin{itemize}
	\item Basics of cryptography
	\item Services provided by PGP
	\item How PGP works
\end{itemize}


\nocite{*}
\printbibliography

\end{document}


\tableofcontents
\clearpage

% INTRODUCTION
\section{Introduction}

Cloud storage is ubiquitous and widely used for services such as backups or
outsourcing data to reduce operational costs. However, these remote servers
cannot be trusted, because administrators, or intruders with root rights, have
full access to the server and consequently to the plaintext data. Thus, to
store sensitive data in a secure way on an untrusted server, the data has to be
encrypted. Simple encryption is however a hindrance to search capabilities for
the data owner. A trivial but \emph{impractical} solution to re-enable
searching functionality is to download the whole database, decrypt it locally,
and then search for the desired results in the plaintext data.

\textbf{Searchable symmetric encryption (SSE)} is a cryptographic primitive
addressing encrypted search. It allows a client to encrypt data in such a
way that it can later generate search tokens to send as queries to a storage
server. Given a token, the server can search over the encrypted data and return
the appropriate encrypted files.
\clearpage

\section{Preliminaries}

\begin{definition}[Symmetric Key Encryption]
A private-key encryption scheme is a set of three polynomial-time algorithms
\(\mathsf{SKE = (Gen, Enc, Dec)}\) such that:
\begin{enumerate}
\item The \emph{key-generation algorithm} \(\mathsf{Gen}\) is a probabilistic
	algorithm that outputs a secret key \(K\).
\item The \emph{encryption algorithm} \(\mathsf{Enc}\) takes as input a key
	\(K\) and a message \(m\) and returns a ciphertext \(c\).
\item The \emph{decryption algorithm} \(\mathsf{Dec}\) takes as input a key
	\(K\) and a ciphertext \(c\) and returns \(m\) if \(K\) was the key
	under which \(c\) was produced.
\end{enumerate}
\end{definition}


\begin{definition}[Pseudorandom Function]
A polynomial-time computable function that cannot be distinguished from random
functions by any probabilistic polynomial-time adversary.

\noindent A funtion \(F \colon \{0,1\}^n \times \{0,1\}^s \rightarrow \{0,1\}^m\)
is a PRF if
\begin{itemize}
\item Given a key \(K\in\{0,1\}^s\) and an input \(X\in\{0,1\}^n\) there is an
	\emph{efficient} algorithm to compute \(F_K(X) = F(X, K)\).
\item \(F_K(i)\) is indistinguishable from a random function, that is,
	given any \(x_1,\dotsc,x_m\), \(F_K(x_1),\dotsc,F_K(x_m)\), no
	adversary can predict \(F_K(x_{m+1})\) for any \(x_{m+1}\).
\end{itemize}

\end{definition}


\begin{definition}[Homomorphic Encryption]
An operation performed on a set of ciphertexts such that decrypting
the result of the operation is the same as the result of some operation
performed on the plaintexts.
\end{definition}
\clearpage


\section{Definitions}

Searchable encryption allows a client to encrypt data in such a way that
it can later generate search tokens to send as queries to a storage
server. Given a search token, the server can search over the encrypted data
and return the appropriate encrypted files.

The data can be viewed as a sequence of \(n\) files
\(\mathbf{f} = (f_1, \dotsc, f_n)\), where file \(f_i\) is a sequence of words
\((w_1, \dotsc, w_m)\) from a universe \(W\). Each file
has a unique identifier \(\mathsf{id}(f_i)\). The data is dynamic, so at
any time a file may be added or removed. The files can be any type of data as
long as there exists an efficient algorithm that maps each document to a
file of keywords from \(W\). Given a
keyword \(w\), \(\mathbf{f}_w\) denotes the set of files in \(\mathbf{f}\)
that contain \(w\). If \(\mathbf{c} = (c_1 ,\dotsc, c_n)\) is a set of
encryptions of the files in \(\mathbf{f}\), then \(\mathbf{c}_w\) refers
to the ciphertexts that are encryptions of the files in \(\mathbf{f}_w\).

\begin{definition}[Dynamic SSE]
A dynamic index-based SSE scheme is a set of nine polynomial-time algorithms
\(\mathsf{SSE = (Gen,}\) \(\mathsf{Enc,}\) \(\mathsf{SrchToken,}\)
\(\mathsf{AddToken,}\) \(\mathsf{DelToken,}\) \(\mathsf{Search,}\)
\(\mathsf{Add,}\) \(\mathsf{Del,}\) \(\mathsf{Dec})\) such that:

\begin{enumerate}
\item \(K \leftarrow \mathsf{Gen}(1^k)\): is a probabilistic algorithm that
	takes as input a security parameter \(k\) and outputs a secret key
	\(K\).

\item \((\gamma, \mathbf{c}) \leftarrow \mathsf{Enc}(K, \mathbf{f})\): is a
	probabilistic algorithm that takes as input a secret key \(K\) and
	a sequence of files \(\mathbf{f}\). It outputs an encrypted index
	\(\gamma\), and a sequence of ciphertexts \(\mathbf{c}\).

\item \(\tau_s \leftarrow \mathsf{SrchToken}(K, w)\): is a (possibly
	probabilistic) algorithm that takes as input a secret key \(K\) and a
	keyword \(w\). It outputs a search token \(\tau_s\).

\item \((\tau_a, c_f) \leftarrow \mathsf{AddToken}(K, f)\): is a (possibly
	probabilistic) algorithm that takes as input a secret key \(K\) and a
	file \(f\). It outputs an add token \(\tau_a\) and a ciphertext
	\(c_f\).

\item \(\tau_d \leftarrow \mathsf{DelToken}(K, f)\): is a (possibly
	probabilistic) algorithm that takes as input a secret key \(K\) and a
	file \(f\). It outputs a delete token \(\tau_d\).

\item \(\mathbf{I}_w \coloneqq \mathsf{Search}(\gamma, \mathbf{c}, \tau_s)\):
	is a deterministic algorithm that takes as input an encrypted index
	\(\gamma\), a sequence of ciphertexts \(\mathbf{c}\) and a search
	token \(\tau_s\). It outputs a sequence of identifiers \(\mathbf{I}_w
	\subseteq \mathbf{c}\).

\item \((\gamma', \mathbf{c}') \coloneqq \mathsf{Add}(\gamma, \mathbf{c},
	\tau_a, c)\): is a deterministic algorithm that takes as input an
	encrypted index \(\gamma\), a sequence of ciphertexts \(\mathbf{c}\),
	an add token \(\tau_a\) and a ciphertext \(c\). It outputs a new
	encrypted index \(\gamma'\) and sequence of ciphertexts
	\(\mathbf{c}'\).

\item \((\gamma', \mathbf{c}') \coloneqq \mathsf{Del}(\gamma, \mathbf{c},
	\tau_d)\): is a deterministic algorithm that takes as input an
	encrypted index \(\gamma\), a sequence of ciphertexts \(\mathbf{c}\),
	and a delete token \(\tau_d\). It outputs a new encrypted index
	\(\gamma'\) and sequence of ciphertexts \(\mathbf{c}'\).

\item \(f \coloneqq \mathsf{Dec}(K, c)\): is a deterministic algorithm
	that takes as input a secret key \(K\) and a ciphertext \(c\) and
	outputs a file \(f\).
\end{enumerate}

A dynamic SSE scheme is correct if for all \(k \in \mathbb{N}\), for all
keys \(K\) generated by \(\mathsf{Gen}(1^k)\), for all \(\mathbf{f}\), for all
\((\gamma, \mathbf{c})\) output by \(\mathsf{Enc}(K, \mathbf{f})\), and
for all sequences of add, delete or search operations on \(\gamma\), search
always returns the correct set of indices.
\end{definition}
\clearpage


\section{Construction}

The dynamic SSE scheme is an extension of the non-dynamic SSE-1 construction,
which is based on the inverted index data structure.

\subsection{SSE-1 construction}

The scheme makes use of a symmetric-key encryption scheme
\(\mathsf{SKE = (Gen,}\) \(\mathsf{Enc,}\) \(\mathsf{Dec)}\),
two pseudorandom functions \(F\) and \(G\),
an array \(\mathtt{A}_s\) referred to as the \emph{search array} and a
dictionary \(\mathtt{T}_s\) referred to as the \emph{search table}.

To encrypt a collection of files \(\mathbf{f}\), the scheme constructs
for each keyword \(w \in W\) a list \(\mathtt{L}_w\) composed of
\(\#\mathbf{f}_w\) nodes
\((\mathtt{N}_1, \dotsc, \mathtt{N}_{\#\mathbf{f}_w})\) that are stored at
random locations in the search array \(\mathtt{A}_s\). The node
\(\mathtt{N}_i\) is defined as
\(\mathtt{N}_i = \langle \mathsf{id, addr}_s(\mathtt{N}_{i+1}) \rangle\),
where \(\mathsf{id}\) is the unique file identifier of a file that contains
\(w\) and \(\mathsf{addr}_s(\mathtt{N})\) denotes the location of
node \(\mathtt{N}\) in \(\mathtt{A}_s\).
The unused cells in \(\mathtt{A}_s\) can be padded with random strings of
appropriate length.

For each keyword \(w\), a pointer to the head of \(\mathtt{L}_w\) is then
inserted into the search table \(\mathtt{T}_s\) under search key
\(F_{K_1}(w)\), where \(K_1\) is the key to the PRF \(F\). Each list
is then encrypted using \(\mathsf{SKE}\) under a key generated
as \(G_{K_2}(w)\), where \(K_2\) is the key to the PRF \(G\).

To search for a keyword \(w\), the client sends the values \(F_{K_1}(w)\) and
\(G_{K_2}(w)\). The server then uses \(F_{K_1}(w)\) with \(\mathtt{T}_s\)
to recover the pointer to the head of \(\mathtt{L}_w\).
\(G_{K_2}(w)\) is then used to decrypt the list and recover the identifiers
of the files that contain \(w\). If \(\mathtt{T}\) supports
\(O(1)\) lookups (for example, using a hash table),
the total search time for the server is linear in
\(\#\mathbf{f}_w\), which is optimal.
\clearpage

\subsection{Dynamic SSE}

The limitations of SSE-1 is that it is not explicitly dynamic.

The difficulty is that the addition, deletion or modification of a
file requires the server to add, delete or modify nodes in the
encrypted lists stored in \(\mathtt{A}_s\).
This is difficult for the server to do since:

\begin{enumerate}
\item upon deletion of a file \(f\), it does not know where
	(in \(\mathtt{A}\)) the nodes corresponding to \(f\) are stored.
\item upon insertion or deletion of a node from a list, it cannot
	modify the pointer of the previous node since it is encrypted.
\item upon addition of a node, it does not know which locations in
	\(\mathtt{A}_s\) are free.
\end{enumerate}

At a high level, these limitations are addressed as follows:


\begin{enumerate}
\item Add an extra (encrypted) data structure
	\(\mathtt{A}_d\) called the \emph{deletion array} that
	stores for each file \(f\) a list \(\mathtt{L}_f\)
	of nodes that point to the nodes in \(\mathtt{A}_s\) that should be
	deleted if file \(f\) is ever removed. So every node in the
	search array has a corresponding node in the deletion array and
	every node in the deletion array points to a node in the search array.
\item Encrypt the pointers stored in a node with
	a homomorphic encryption scheme. By providing the server with an
	encryption of an appropriate value, it can then modify the pointer
	without ever having to decrypt the node.
\item To keep track of which locations in \(\mathtt{A}_s\)
	are free, a free list is maintained to add and manage extra space
	the server uses to add new nodes.
\end{enumerate}
\clearpage

% CONCLUSION
\section{Conclusion}
Searchable encryption has become an important problem in security and
cryptography primarily due to the wide proliferation of sensitive data
in open information and communication infrastructures like Amazon S3 and
Microsoft Azure Storage. For example, legislation around the world
stipulates that electronic health records (EHRs) should be encrypted,
which immediately raises the question of how to search EHRs efficiently
and securely.

A lot of developments have taken place in this line of study over the
past decade and research continues in the uncharted areas that are not
covered by the techniques currently in existence.

% References
\clearpage
\nocite{*}
\printbibliography

\end{document}
