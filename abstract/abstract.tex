\documentclass[12pt,a4paper]{article}
\usepackage[english]{babel}
\usepackage[backend=biber]{biblatex}
\usepackage{csquotes}
\usepackage{hyperref}

\addbibresource{ref.bib}

\hypersetup{
	hidelinks,
	pdftitle={Dynamic Searchable Symmetric Encryption},
	pdfauthor={Arpan Kapoor},
	pdfsubject={Seminar Abstract},
	pdfkeywords={seminar,abstract}
}

\begin{document}

\title{Dynamic Searchable Symmetric Encryption}
\author{Arpan Kapoor\\
	\href{mailto:arpan_b120555cs@nitc.ac.in}
	{\nolinkurl{arpan_b120555cs@nitc.ac.in}}
}
\date{September 20, 2015}

\maketitle

\begin{center}
	\textbf{Domain:} Computer Security
\end{center}

\begin{abstract}
Searchable symmetric encryption (SSE) allows a client to encrypt its data in
such a way that this data can still be searched. The most immediate application
of SSE is to cloud storage, where it enables a client to securely outsource its
data to an untrusted cloud provider without sacrificing the ability to search
over it.

A practical SSE scheme should (at a minimum) satisfy the following properties:
sublinear search time, security against adaptive chosen-keyword attacks,
compact indexes and the ability to add and delete files efficiently.
A SSE scheme to satisfy all the properties outlined above is discussed.
\end{abstract}

\nocite{*}
\printbibliography
\end{document}
